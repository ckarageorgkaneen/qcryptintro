The EPR paradox, although, initially designed to demonstrate quantum mechanics' incompleteness 
as a physical theory, subsequently came to show how it defies classical intuition.
Einstein, Podolsky \& Rosen mainly relied on two classical principles: locality and realism,
stating that distant objects cannot have direct influence on one another and that there
exists a reality that is independent of an observer or measurer. Quantum cryptography
violates both by adhering to the following principles:
\begin{center}
  \begin{enumerate}
    \item The polarization of a photon cannot be measured in non-compatible bases
          at the same time (e.g. bases `$\bm{+}$' and `$\bm{\times}$' ).
    \item Individual quantum processes cannot be distinctly described.
    \item Measuring a quantum system always disturbs it.
    \item It is impossible to duplicate unknown quantum states.
  \end{enumerate}
\end{center}
The third principle plays an essential role in ensuring that the most valued property of
quantum cryptography, its security, is preserved. It implies, as has already been examined,
that Eve cannot eavesdrop on a message sent between Bob and Alice, without altering it,
and therefore being exposed. And she cannot conceal herself due to the fourth principle.
The E91 protocol further exploits quantum entanglement in which although no definite
conclusions can be made about the state of each of the particle before measurement, the
state of both particles is well defined, post-measurement.
Quantum entanglement refers to the inability to define the quantum state of one object
without reference to the quantum state of another object, entangled to the first.
Up until now we have seen the BB84 protocol of which the E91 protocol is a modification.
The former used four quantum states, with each pair making up a different base.
Alice sent particles to Bob in one of the four states, and Bob randomly selected bases in
which to measure the particles. As shall be discussed, E91 uses six and also
EPR states are taken into consideration.
