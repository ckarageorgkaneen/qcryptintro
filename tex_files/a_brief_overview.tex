Quantum cryptography deals with the exploitation of quantum mechanical properties of elementary particles, 
such as photons, for accomplishing cryptographic communication tasks, such as the encryption and decryption of 
information at the ends of both the sender and the receiver, named Alice and Bob, respectively, by ensuring the
confidentiality of the transmission.
The advantage quantum mechanical methods over classical ones mainly has to do with the innate spontaneity of certain 
of the former's unique properties, whose realization render the latter extremely time-consuming, in comparison. 
For instance, the no-cloning theorem, combined with the inability of reading data encoded in a quantum state
without changing it, make it possible to detect eavesdropping attempts, aka Eves.
Despite there being many advances in methods and protocols over the years, such as quantum coin-flipping, 
quantum commitment, the bounded and noisy quantum-storage models, position-based quantum cryptography, device-independent quantum
cryptography, post-quantum cryptography, etc., this paper focuses on the most important to-date and widely-applied secure 
communication scheme, called quantum key distribution (QKD). It involves two parties being able to produce a shared random secret
key that can be used to encrypt (``lock") and decrypt (``unlock") the data bits to be communicated between two parties. 
In classical systems, cryptographical keys are generated using mathematical algorithms that are very hard (though not impossible) to break. 
Whereas, quantum cryptography uses a method of key distribution that relies on the laws of quantum physics in order to create a key. 
Although not completely hacker-proof, quantum cryptography offers huge advantages over traditional methods. 
It has also been hypothesized, though not at the time of writing fully proven, that currently used popular 
public-key encryption and signature methods, such as elliptic−curve cryptography (ECC) and RSA, can be broken 
by quantum adversaries.
